\documentclass[a4paper,10pt]{article}

\usepackage[utf8]{inputenc}
\usepackage{t1enc}
\usepackage[spanish]{babel}
\usepackage[pdftex,usenames,dvipsnames]{color}
\usepackage[pdftex]{graphicx}
\usepackage{amsmath}
\usepackage{amsfonts}
\usepackage{amssymb}
\usepackage{listings}
\lstset{language=C}
\lstset{showstringspaces=false}
\lstset{basicstyle=\ttfamily,}

\begin{document}

\begin{titlepage}
        \thispagestyle{empty}
        \begin{center}
                \includegraphics{./images/itba.jpg}
                \vfill
                \Huge{Arquitectura de las Computadoras}\\
                \vspace{1cm}
                \huge{Manual de usuario: Arqvengers OS}\\
        \end{center}
        \vspace{2cm}
        \large{
                \begin{tabular}{lcrc}
                        \textbf{Alvaro Crespo} & & 50758 & \ \ \texttt{acrespo@alu.itba.edu.ar}\\
                        \textbf{Juan Pablo Civile} & & 50453 & \ \ \texttt{jcivile@alu.itba.edu.ar}\\
                        \textbf{Dario Susnisky} & & 50592 & \ \ \texttt{dsusnisk@alu.itba.edu.ar}\\
                        \\ 
                \end{tabular}
        }
        \vfill
        \flushright{1 de Junio del 2011}
\end{titlepage}

\setcounter{page}{1}

\tableofcontents
\newpage

\section{Resumen}
    Este manual tiene el proposito de hacer de guia para que cualquier usuario, principiante o no pueda utilizar el sistema operativo Arqvengers OS en su total funcionalidad. Esto incluye inicializacion del sitema, manejo general del sistema operativo y la ejecucion de los comandos proveidos por el OS.

\section{Inicializacion del sistema}
    Para poder inicializar el sistema de forma correcta simplemente basta con asegurarse que la imagen del OS este cargada en un dispositivo booteable (CD, disquette o pen drive como ejemplos) y que la computadora este configurada para incializarse desde tal dispositivo. Una vez aseguradas estas condiciones Arqvengers OS cargara de forma correcta y sin ninguna necesidad de otras especificaciones por parte del usuario.

\section{Manejo general del sistema operativo}
    Arqvengers OS provee como interface principal un interprete de comandos (Shell) la cual se puede visualizar al iniciar el sistema. La funcionalidad basica de la misma implica tipear el comando que se desee y luego ejecutarlo con la tecla enter. El manejador de teclado reconoce teclas que representen letras, numeros y simbolos, y tambien las teclas modificadoras (shift, alt, ctrl y Bloq Mayus). Las flechas de derecha, izquierda, inicio y fin pueden ser utilizadas para recorrer la linea escrita por el usuario a gusto.
    La shell de Arquvengers OS tambien soporta manejo de colores para ciertas funcionalidades y le provee al usuario las opciones de utilizar autocompletado, historial y el uso de varias shells lo cual sera explicado a continuacion:

    \subsection{Autocompletado}
        La shell de Arqvengers OS permite autocompletar comandos para facilidad del usuario. Eso significa que con presionar la tecla tab en cualquier momento, se le completara al usuario con el comando comenzando con lo que el ya escribio. En caso de que haya un solo comando comenzando de esa manera, este es el unico que aparecera. En caso contrario, se le mostrara al usuario una lista con las opciones. Como ejemplo podemos poner que si escribimos la tecla 'h' y luego tocamos tab automaticamente aparecera en pantalla la palabra help, ya que es el único comando que comienza con la letra 'h'.
    \subsection{Historial}
        Se le permite al usuario acceder al historial de los comandos ejecutados. Esto se realiza presionando la flecha que apunta para arriba en el teclado. Así si ejecute primero el comando help y luego el comando fortune y presiono la flecha de arriba, se vera la palabra fortune sobre la linea actual. Si se vuelve a presionar la flecha, se vera help.
    \subsection{Multiples Shells}
        Arqvengers OS provee multiples shells permitiendo así al usuario realizar trabajo en paralelo y mantener estados en determinadas shells. Para navegar entre shell y shell basta con presionar alt + f1 para ir al a shell numero 1 y asi para cada shell. Hay un maximo de 4 shells por lo cual solo es valido presionar hasta alt + f4.

\vspace{1cm}

Ademas, Arqvengers OS permite al usuario reiniciar el sistema facilmente combinando las teclas ctrl + alt + supr.
Por último es importante recalcar que el uso del comando Help le puede ayudar al usuario en cualquier momento en la shell los comandos disponibles.

\section{Comandos proveidos por Arqvengers OS}
    A continuación se presenta una lista de los comandos proveidos por Arqvengers OS y la explicación de funcionamiento:

    \subsection{echo}
       El comando echo permite imprimir los argumentos que se le pasen a pantalla.
       El funcionamiento es sencillo: echo argumento1 argumento2 ... argumentoN.
       El resultado de tal operación es que en pantalla se vera "argumento1 argumento2 ... argumentoN".
       Veamos un ejemplo:
       "echo hola mundo" refleja en la siguiente linea de la pantalla "hola mundo". 
    \subsection{man}
        El comando man permite ver la pagina de manual de uso de cualquier comando.
        El uso es: "man nombreComando".
        Así, "man echo" imprimira en pantalla el manual de como utilizar el comando echo.
    \subsection{help}
        El comando help permite ver todos los comandos disponibles. Su uso implica simplemente tipear la palabra help.
    \subsection{sudoku}
        Este comando permite jugar un juego de Sudoku.
        Una vez inicializado el comando Sudoku se vera una pagina introductoria explicando como jugar, la cual puede saltearse tocando la tecla Enter.
        Una vez dentro del juego uno puede moverse por el tablero con las flechas y agregar los numeros con los numeros. Tratar de poner el numero 0 borrara la casilla excepto que esta haya estado llena desde el estado inicial. En caso de que el número sea incorrecto la celda se pintara de rojo.
        Es posible salir del juego de sudoku presionando la tecla 'q' en cualquier momento.
    \subsection{calc}
        El comando calc permite realizar cuentas matematicas entre numeros enteros.
        Una vez dentro del programa calc pueden hacerse estas cuentas simplemente escribiendo calculos con el formato X op Y (siendo X e Y numeros y op siendo un simbolo entre +, -, * y /).
        Dentro de calc tambien se puede usar el comando help para pedir ayuda sobre el uso de la calculadora.
        Se puede salir del programa ejecutando quit o exit.
    \subsection{getCPUSpeed}
        El comando getCPUSpeed permite obtener la velocidad del CPU. Su uso implica simplemente tipear el comando getCPUSpeed.
    \subsection{fortune}
        El comando fortune permite obtener frases que contentran situaciones humoristicas o proveeran de puro conocimiento, esto, depende puramente del azar. Su uso implica simplemente tipear el comando fortune.
    \subsection{date}
        El comando date permite obtener la fecha y hora actual. Su uso implica simplemente tipear el comando date.
\vspace{2cm}

Desde los Arqvengers Headquartes esperamos con sinceridad que este manual le haya sido util. ¡Disfrute de Arqvengers OS!

\end{document}
