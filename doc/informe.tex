\documentclass[a4paper,10pt]{article}

\usepackage[utf8]{inputenc}
\usepackage{t1enc}
\usepackage[spanish]{babel}
\usepackage[pdftex,usenames,dvipsnames]{color}
\usepackage[pdftex]{graphicx}
\usepackage{amsmath}
\usepackage{amsfonts}
\usepackage{amssymb}
\usepackage{listings}
\lstset{language=C}
\lstset{showstringspaces=false}
\lstset{basicstyle=\ttfamily,}

\begin{document}


\renewcommand{\lstlistingname}{C\'odigo Fuente}
\lstloadlanguages{Octave} 
\lstdefinelanguage{MyPseudoCode}[]{Octave}{
	deletekeywords={beta,det},
	morekeywords={repmat}
} 
\lstset{
	language=MyPseudoCode,
	stringstyle=\ttfamily,
	showstringspaces = false,
	basicstyle=\footnotesize\ttfamily,
	commentstyle=\color{gray},
	keywordstyle=\bfseries,
	numbers=left,
	numberstyle=\ttfamily\footnotesize,
	stepnumber=1,                   
	framexleftmargin=0.20cm,
	numbersep=0.37cm,              
	backgroundcolor=\color{white},
	showspaces=false,
	showtabs=false,
	frame=l,
	tabsize=4,
	captionpos=b,               
	breaklines=true,             
	breakatwhitespace=false,      
	mathescape=true
}
\begin{titlepage}
        \thispagestyle{empty}
        \begin{center}
                \includegraphics{./images/itba.jpg}
                \vfill
                \Huge{Sistemas Operativos}\\
                \vspace{1cm}
                \huge{Trabajo Práctico Especial}\\
        \end{center}
        \vspace{2cm}
        \large{
                \begin{tabular}{lcrc}
                        \textbf{Alvaro Crespo} & & 50758 & \ \ \texttt{acrespo@alu.itba.edu.ar}\\
                        \textbf{Juan Pablo Civile} & & 50453 & \ \ \texttt{jcivile@alu.itba.edu.ar}\\
                        \textbf{Darío Susnisky} & & 50592 & \ \ \texttt{dsusnisk@alu.itba.edu.ar}\\
                        \\ 
                \end{tabular}
        }
        \vfill
        \flushright{24 de Octubre del 2011}
\end{titlepage}

\setcounter{page}{1}

\tableofcontents
\newpage
\section{Introducción}
El trabajo práctico consistía en la extensión del sistema operativo armado en la materia arquitectura de las 
computadoras. Este debia ser modificado de modo tal que esta nueva versión soporte procesamiento multitareas
 (inclusive con más de una técnica de 	extit{scheduling}), soporte de acceso a disco rígido y un \textit{file system}
 no tan básico con varios detalles interesantes (soporte de usuarios y grupos, \textit{FIFO's}, entre otros).

\newpage
\section{Breve resumen de la vieja versión de Arqvengers OS}
El trabajo práctico fue comenzado tomando como base el trabajo hecho en la materia Arquitectura de las Computadoras.
 Este trabajo contaba con un sistema operativo booteable que contaba con soporte para \textit{drivers} de
  teclado y de video facilitando varias consolas en donde podían ejecutarse distintos comandos. 
  A un nivel más bajo, contabamos con una rudimentaria e incompleta librería de C así como una interfaz para realizar
  llamadas a sistema.
\newpage
\section{Procesos}

\subsection{Estructura de un proceso}
La estructura de proceso contiene toda la informacion necesaria para ejecutar el proceso y atender llamadas al sistema.
Basicamente la informacion que contiene es:
\begin{itemize}
\item Contexto de ejecucion
\item Padre
\item Usuario y grupo
\item Archivos abiertos
\item Estado de scheduling
\end{itemize}

El contexto de ejecucion es la parte fundamental de la estructura.
Este contiene la memoria que es el stack y el heap del proceso.
Tambien contiene una referencia al punto de entrada del proceso, y los argumentos pasados al momento de creacion.

Para permitir la interaccion con el file system, el proceso tiene una tabla de que archivos tiene abiertos.
Ademas, el usuario que lo ejecuto, que es necesario para verificar permisos.

    \subsection{Scheduling}
    Como primera estrategia de \textit{scheduling} se decidió implementar una simple estrategia de \textit{Round Robin}.
    Esto es, los períodos de tiempo de uso de CPU (\textit{time slices}) son asignados a los procesos en porciones 
    equivalentes y en orden circular, sin hacer uso de un sistema de prioridades. Este estrategia, además de ser muy 
    fácil de implementar, asegura la provisión de tiempo de CPU a todos los procesos (\textit{starvation free}).

    Para nuestra segunda estrategia, y siguiendo con la consigna del trabajo, implementamos un sistema basado en 
    prioridades. 

\subsection{Multiples Terminales}
Para el soporte de varias terminales, optamos por crear un proceso \textit{tty}, como mediador entre las \textit{shells} y el teclado y pantalla.
Este proceso recibe los pedidos de IO de los procesos que deseen escribir a pantalla o leer de teclado.

Una de sus tareas es tener en cuenta que proceso es activo, y por lo tanto, si puede o no leer del teclado.
Si un proceso no activo intenta leer de teclado, se lo bloquea hasta que este estado cambie, o se mate al proceso.
De la misma manera, para escribir a pantalla, se tiene el concepto de terminal activa.
Es responsabilidad de \textit{tty} asegurar que solo se vea el output de la terminal activa, y que ante un cambio de terminal se redibuje la pantalla.

Una de las motivaciones mas importantes para hacer de esto un proceso dedicado, es poder interpretar el input de teclado aun mientras el proceso activo no pida entrada.
Es decir, es posible cambiar entre terminales cuando ningun proceso esta pidiendo input, ya que esta operacion la hace \textit{tty}.

Una particularidad muy importante de \textit{tty} es que es uno de los dos procesos de \textit{kernel}.
Al decir proceo de kernel, significa que corre en el contexto de memoria del kernel, y por lo tanto, puede hacer acceso directo a operaciones del kernel.
Esto resulto ser necesario y muy valioso, por que permite a este proceso acceder a informacion de los procesos, y bloquearlos en caso de ser necesario.
A pesar de esto, en todo otro aspecto se lo trata como un proceso normal.

Una vez inicializado \textit{tty} este crea los procesos de \textit{login}, cada uno en su propia terminal.
Este proceso se encarga de verificar las credenciales del usuario, y en caso de ser correctas, ejecutar el shell para que el usuario pueda utilizar el sistema.
Entonces, cada shell es un proceso independiente, en su propia terminal.

\newpage
\section{Driver Ata}
    Se comenzó a implementar un \textit{driver} ATA en modo PIO. Esto significaba que los accesos a disco iban a ser
    por medio de puertos de entrada/salida. Esto hace que el tiempo de ejecución sea considerable pero era un estandar
    confiable y sencillo de implementar.
    El estandar permite acceder a dos discos simultaneamente pero nosotros no vimos la necesidad de implementar el 
    acceso a más de uno.
    
    El primer paso era detectar el disco y obtener ciertos datos del disco en sí. Por suerte, GRUB provee una estructura
    con varios datos del sistema, entre ellos, datos del disco. Trás leer las especificaciones de \textit{multiboot}
    fue relativamente sencillo realizar este trabajo.

    A la hora de elegir el modo de direccionamiento nos encontramos con tres opciones. CHS se vió descartada de forma
    inmediata dado su estado de obsoleta. Luego existian las opciones de LBA 28 y LBA 48 (sus números representan
    la cantidad de bits relevantes utilizados para la dirección). LBA 28 también estaba obsoleto para ciertos discos
    pero era más rapido, más sencillo de implementar y dejaba satisfechas las necesidades de este trabajo.

    Las unidades minimas de lectura y escritura en nuestra implementación son de a sectores, pudiendo ser de a
    varios a la vez. Es un paso escencial poner los datos correctos en los puertos necesarios previo a los accesos, pero
    es aún más interesante la lectura del estado del disco. Esto es necesario para detectar errores en el disco
    y para saber si el disco esta listo para recibir o enviar datos. Es posible que haya que esperar a que el disco
    este listo antes de realizar alguna operación. Hay dos maneras de realizar esta espera, se puede esperar a que una IRQ
    indique que el disco esta listo o se puede consultar el estado del disco hasta que este este listo (\textit{polling}).
    Hacer \textit{polling} puede hacer que se pierda tiempo en un sistema multitareas, pero por otro lado una sola
    consulta de \textit{polling} es más rápida que esperar a una IRQ e implementar \textit{polling} es mucho más sencillo.
    Por esta última razón fue elegida la técnica de \textit{polling}.
\newpage
\section{File System}
    
Una vez implementado el \textit{driver} ATA era posible comenzar a implementar nuestro \textit{file system}.
Dado que nuestra intención era realizar un \textit{file system} persistente debiamos contar con ciertas estructuras
y ciertas especificaciones de como iba a estar representado el mismo en el disco. Luego de hacer un relevamiento
en el tópico decidimos que el implementar un sistema de tipo Ext2 abarcaba todas las caracteristicas que deseabamos
en nuestro sistema. 

Si bien es posible que un sistema Ext2 exceda los requerimientos del trabajo práctico, decidimos que era más seguro seguir este estandar. 
Las herramientas provistas en la distribuciones de Linux para la manipulacion de ext2 probaron ser el mayor beneficio de esta eleccion.
Contar con \textit{fsck} permitio asegurar el correcto funcionamiento de nuestra implementacion.

Una vez terminado el esqueleto que significaba el \textit{file system} se pudo profundizar en el mismo agregando
nuevas funcionalidades, particularmente vamos a hacer hincapie en las consecuencias de aperturas y manejo de
archivos, directorios, FIFO's, links simbolicos, \textit{current working
directory}, permisos del sistema y usuarios y grupos.

A continuación se presentan secciones detalladas sobre estos puntos.

\subsection{Sistema Ext2}
El sistema Ext2 provee un estandar de como guardar la información en el disco.
Existen varias versiones de ext2, y una serie de extensiones al mismo, pero por simpliciadad nosotros optamos por la version 0.
Esta es la primera, y mas simple version de ext2.

Sobre la division basica de sectores de disco, ext2 crea el concepto de bloques.
Cada bloque, o \textit{Block}, se compone de uno o mas sectores fisicamente contiguos.
Todos los bloques dentro de una particion ext2 son del mismo tamaño, y tienen un indentificador unico.

Sobre los bloques se construye la estructura en disco, compuesta de un \textit{Superblock} y varios \textit{Block Groups}.
A su vez, cada \textit{Block Group} contiene varios \textit{Inodes}.

\subsubsection{Superblock}
El \textit{Superblock} contiene la informacion basica y necesaria sobre el file system.
Esto es, el tamaño de bloque, la cantidad de bloques en la particion, y otras estadisticas utiles de uso.
Como primer paso en montar el file system, se debe cargar esta estructura, que siempre esta ubicada a partir del byte 1024 de la particion.
Al ser utilizada frecuentemente, esta informacion una vez leida se la mantiene siempre en memoria.

\subsubsection{Inodes}
Un inode es la representacion fisica de un archivo.
O sea, un nodo en el file system.
Pero no debe confundirse con una entrada en el arbol de directorios, que como ya veremos, son independientes de los \textit{inodes}.

Cada inode contiene informacion basica sobre si mismo.
Esto son tipo, permisos, tamaño, fecha de creacion, etc...
Por supuesto, tambien contiene referencias a los bloques donde se almacena el contenido del inode.

\subsubsection{Block Groups}
En busqueda de eficiencia y organizacion, ext2 divide los bloques de la particion en varios grupos.
Estos grupos tienen un numero igual de bloques y de inodes.

El objetivo principal de creacion de estos, es reducir la fragmentacion. 
Permiten buscar bloques disponibles para uso dentro del mismo grupo en el que se encuentra un inode, asi teniendo toda la informacion en bloques lo mas cercanos posibles.

En particular, estos contienen:
\begin{itemize}
\item Bitmap de uso de bloques
\item Bitmap de uso de inodes
\item Tabla de inodes
\item Bloques de datos
\end{itemize}

La ubicacion de esta informacion, y otros datos sobre el uso de un dado grupo, se encuentra en la \textit{Block Group Descriptor Table}.
Esta se encuentra en el bloque inmediatamente posterior al \textit{Superblock}.
En ella se almacenan muchos datos utilizados con el objetivo de poder verificar el correcto estado del file system.

\subsubsection{Entradas de directorio}
ext2 define como se almacena el contenido de un directorio.
Se hace de una manera muy simple en forma de \textit{Linked List}.
Cada entrada contiene un nombre, y un numero de inode al que hace referencia.
Como consecuencia, cada inode puede ser referencia con mas de un nombre, creando lo conocido como \textit{hard links}.

\subsection{Virtual file system}

    \subsection{Apertura y manejo de archivos}
    En una primera instancia nos encargamos de adaptar nuestras llamadas a sistema y nuestra librería estandar
    para que puedan desarrollar las nuevas funcionalidades de nuestro sistema operativo. Las llamadas a \textit{read} y
    \textit{write} podían ahora escribir en archivos, así como las llamadas a \textit{open} y \textit{close} se hicieron
    necesarias cuando antes no lo fueron.
    Al hacer una llamada a \textit{open()}, esta actualizaba tanto a nivel kernel como a nivel proceso que cierto
    archivo habia sido abierto. Como ya fue dicho, un proceso contaba con una tabla de \textit{file descriptors}. 
    Estos, apuntan al \textit{inode} siendo manipulado y además cuentan con punteros a función que indican como 
    deben comportarse lecturas y escrituras (entre otras cosas) dependiendo el tipo de archivo 
    (podría bien ser un archivo regular o un link simbolico por ejemplo). Esto permitió que las implementaciones 
    a \textit{read()}, \textit{write()} y \textit{close()} sean bastante sencillas. 
    
    Estas llamadas a sistema fueron siendo extendidas a medida que la funcionalidad del sistema operativo crecía. Como
    será visto en las proximas secciones, se escribieron funciones adecuadas para la manipulación de cada tipo de archivo.
    
    Para realizar todas estas llamadas a sistema se explotó en su máximo potencial las prestaciones del vfs.
    
    Por último, es interesante destacar que dado que el usuario se maneja con cadenas de texto que representan 
    \textit{inodes}, fueron armadas funciones que permiten resolver estas cadenas, dividiendo directorios y archivos,
    proveyendo finalmente el \textit{inode} deseado. 

    \subsection{Directorios}
    
    Ya a bajo nivel era posible detectar el tipo de \textit{inode} guardado en el sistema. Un directorio simplemente era
    un inode que en su contenido se guardan referencias hacía otros inodes. Tanto la interfaz del driver Ext2 como la del
    vfs permiten escribir directorios de forma correcta. Cabe destacar que ademas de la referencia al inode, una entrada
    de directorio también guarda el nombre de la entrada que luego verá el usuario.

    \subsection{FIFO's}

    \subsection{Links simbolicos}
    
    La implementación de links simbolicos implicaba la creación de nodos que inclusive en el sistema Ext2 eran marcados
    como links simbolicos. El contenido de los mismos simplemente indicaba el \textit{path} del archivo al que este
    hacia referencia.
    Luego, a la hora de leer un archivo que sea un link simbolico (o al tratar de resolver un \textit{path} cuyo alguno
    de sus directorios era un link simbolico) bastaba con reemplazar el este archivo con el link al que realmente
    apuntaba.

    En esta implementación (al igual que en el resto) tratamos de ser lo más fieles al entorno \textit{UNIX} como 
    era posible.

    \subsection{Current working directory}
        
    Como ya fue mencionado, cada proceso contiene información sobre cual era el \textit{current working directory} en
    el momento en que fue ejecutado. Esto permite que la obtención del \textit{current working directory} sea sencilla
    y manipulable en los momentos necesarios, sobre todo al modificar el árbol del \textit{file system}.

    \subsection{Usuarios y grupos}
    
    \subsection{Permisos del sistema}

    Una vez implementados usuarios y grupos era fácil saber el gid y el uid de un proceso. Dada la forma en que estaba
    implementado nuestro \textit{file system}, Ext2 contaba con lugares especificos para guardar tanto los permisos y el
    uid y gid del archivo. Además, las prestaciones del vfs hacían que estos datos también fuesen fáciles de obtener.

    Una vez conseguidos estos datos, era trivial evaluar en que casos se permitía abrir un archivo, teniendo en cuenta
    que tanto el archivo tenga los permisos adecuados así como el \textit{path} que lo contiene.

\newpage
\section{Comandos provistos}
    Para poder probar y mostrar las nuevas funcionalidades del trabajo hubo que agregar ciertos comandos y
    funcionalidades ejecutables desde la consola. A continuación se presentan algunos casos interesantes.

\newpage
\section{Problemas encontrados}

Durante el desarrollo de este trabajo practico fueron surgiendo diferentes dudas y problemas. El proposito de esta
sección es comentarlos con el fin del aprendizaje.

Un dilema bastante frecuente a lo largo del trabajo practico es lo que nos gusta llamar "gallina o huevo". Al realizar
un desarrollo a bajo nivel que ha comenzado con pocos recursos, era habitual encontrarse con la duda de si convenía
arrancar a implementar la lectura o la escritura de alguna prestación. Es evidente que si se hacía primero la lectura, 
no había forma de probar su funcionamiento ya que no existía la escritura. Lo mismo sucedía a modo inverso. Si bien 
la solución no era complicada (desarrollar todo) era incomodo y molesto a la hora de encontrar errores ya que los mismos
podían encontrarse en cualquier lado.

Varios problemas interesantes surgieron durante la creación del \textit{driver} ATA. A pesar de que se siguieron
rigurosamente los estándares, se encontró que no todos los emuladores funcionaban de forma correcta con el mismo.
Inclusive, detectamos que el driver no se comportaba de manera deseada al tratar con varios sectores a la vez con lo
cual nos vimos forzados a modificar la lógica interna del driver simulando un acceso a multiples sectores a tráves de
varios accesos de a un sector a la vez.

\newpage     
\section{Referencias}

\begin{itemize}
  \item Material provisto por la cátedra
  \item The C programming language - Kernighan y Ritchie
  \item http://invisible-island.net/xterm/ctlseqs/ctlseqs.html
  \item http://webpages.charter.net/danrollins/techhelp/0087.HTM
  \item http://faydoc.tripod.com/cpu/rdtsc.htm
  \item http://stanislavs.org/helppc/
  \item http://www.linux.it/~rubini/docs/ksys/ksys.html
  \item http://wiki.osdev.org
  \item http://wiki.osdev.org/Detecting\_CPU\_Speed
  \item	http://wiki.osdev.org/CMOS\#Accessing\_CMOS\_Registers
  \item http://wiki.osdev.org/Bootable\_CD
  \item http://wiki.osdev.org/Boot\_sequence\#Easy\_Way\_Out
  \item http://wiki.osdev.org/Ext2
  \item http://wiki.osdev.org/IDE
  \item http://wiki.osdev.org/ATA\_PIO\_Mode
  \item http://en.wikipedia.org/wiki/System\_time\#Retrieving\_system\_time
  \item http://en.wikipedia.org/wiki/Calculating\_the\_day\_of\_the\_week
  \item http://cplusplus.com/
  \item http://github.com/esneider/malloc
\end{itemize}
   
\end{document}
