\documentclass[a4paper,10pt]{article}

\usepackage[utf8]{inputenc}
\usepackage{t1enc}
\usepackage[spanish]{babel}
\usepackage[pdftex,usenames,dvipsnames]{color}
\usepackage[pdftex]{graphicx}
\usepackage{amsmath}
\usepackage{amsfonts}
\usepackage{amssymb}
\usepackage{listings}
\lstset{language=C}
\lstset{showstringspaces=false}
\lstset{basicstyle=\ttfamily,}

\begin{document}


\renewcommand{\lstlistingname}{C\'odigo Fuente}
\lstloadlanguages{Octave} 
\lstdefinelanguage{MyPseudoCode}[]{Octave}{
	deletekeywords={beta,det},
	morekeywords={repmat}
} 
\lstset{
	language=MyPseudoCode,
	stringstyle=\ttfamily,
	showstringspaces = false,
	basicstyle=\footnotesize\ttfamily,
	commentstyle=\color{gray},
	keywordstyle=\bfseries,
	numbers=left,
	numberstyle=\ttfamily\footnotesize,
	stepnumber=1,                   
	framexleftmargin=0.20cm,
	numbersep=0.37cm,              
	backgroundcolor=\color{white},
	showspaces=false,
	showtabs=false,
	frame=l,
	tabsize=4,
	captionpos=b,               
	breaklines=true,             
	breakatwhitespace=false,      
	mathescape=true
}
\begin{titlepage}
        \thispagestyle{empty}
        \begin{center}
                \includegraphics{./images/itba.jpg}
                \vfill
                \Huge{Arquitectura de las Computadoras}\\
                \vspace{1cm}
                \huge{Trabajo Práctico Especial}\\
        \end{center}
        \vspace{2cm}
        \large{
                \begin{tabular}{lcrc}
                        \textbf{Alvaro Crespo} & & 50758 & \ \ \texttt{acrespo@alu.itba.edu.ar}\\
                        \textbf{Juan Pablo Civile} & & 50453 & \ \ \texttt{jcivile@alu.itba.edu.ar}\\
                        \textbf{Darío Susnisky} & & 50592 & \ \ \texttt{dsusnisk@alu.itba.edu.ar}\\
                        \\ 
                \end{tabular}
        }
        \vfill
        \flushright{1 de Junio del 2011}
\end{titlepage}

\setcounter{page}{1}

\tableofcontents
\newpage

\section{Diseño del Kernel}
    Las funcionalidades del kernel fueron desarrolladas incrementalmente.
    Inicialmente, optamos por implementar las system calls \textit{read} y \textit{write} de una manera muy sencilla.
    Una vez que teníamos esta versión desarrollada, comenzamos a pensar como interactúan éstas con las aplicaciones.
    Al estar usando \textit{read} y \textit{write} como las define \textit{Linux}, decidimos investigar el I/O de teclado y video en las terminales de sistemas \textit{Unix-compatible}.
    Dado que el objetivo del desarrollo no es diseñar un sistema operativo nuevo, optamos por ser lo más compatibles con \textit{Linux} posible.

    Investigando, aprendimos que \textit{Linux} maneja esta interacción de una manera que emula las viejas terminales de mainframes.
    Esto significa, que hay solo 3 operaciones posibles: \textit{read}, \textit{write} e \textit{ioctl}.
    \textit{Ioctl} es una system call que permite comunicarse con los drivers detrás de un dado archivo de maneras específicas a cada driver.
    Esta es la llamada que permite manipular el I/O para obtener los distintos comportamientos observados en una consola en \textit{Linux}.
    También encontramos que las terminales aceptan las llamadas escape sequences, que permiten realizar operaciones especiales sobre la pantalla, como manipular el color y el cursor.

    En fin, esto derivó en la creación de una emulación de una \textit{tty} a nivel kernel.
    Si bien la emulación es limitada, es en parte compatible con la mayoría de las shells modernas de \textit{Linux}.
    Para esto se crearon los drivers de video y teclado, con los cuales se comunica desde las aplicaciones usando \textit{read}, \textit{write} e \textit{ioctl}.
    Dado que no existen más drivers, \textit{read} y \textit{write} se comunican directamente con los drivers.

    Esta emulación nos permitió gran flexibilidad y aportó un nivel de funcionalidad muy alto para el desarrollo de aplicaciones.
    Por ejemplo, se prestó con facilidad a la implementación de multiples \textit{tty}.

    \subsection{Segmentación y paginación}
        El SO no utiliza ni segmentación ni paginación. 
        Es decir, utilizamos modo flat, sin paginación.
        La configuración de segmentos en modo flat, viene provista por GRUB.
        El mismo crea 2 segmentos, uno de código, y otro de datos.
        Estos ocupan toda la extensión del mapa de memoria.

    \subsection{Interrupciones}
        Con respecto al manejo de interrupciones, se utilizaron las funciones provistas por la cátedra para la inicialización de la \textit{IDT} y
        la carga de de interrupciones.

        \subsubsection{Remapeo del \textit{PIC}}
            Cabe destacar que, debido a que en modo real las interrupciones del \textit{PIC} se solapan con las excepciones, y que así era como el bootloader
            nos dejaba la \textit{IDT}, tuvimos que remapear el \textit{PIC}. Esto es, movimos las interrupciones del \textit{PIC}, las \textit{IRQs}, a otras posiciones de la \textit{IDT}. Por sentido común, y por lo que investigamos, decidimos
            ubicarlas luego de las primeras 32 posiciones que ocupan las excepciones, ocupando así las posiciones 33 a 48. (Tener en cuenta
            que tanto el \textit{PIC} principal como el esclavo manejan 8 interrupciones cada uno).
        
        \subsubsection{Atendiendo interrupciones}
            Al definir los handlers de las interrupciones y las excepciones, que se guardarían en la \textit{IDT}, nos dimos cuenta que los códigos sería muy parecidos,
            por no decir casi idénticos. Por esta razón, hicimos uso de las macros de assembler para ahorrar tiempo y trabajo, a la vez
            contribuyendo a la claridad del código.
            
            Para nuestra comodidad, y siempre pensando en la claridad y prolijidad del código, todos los handlers, tanto de interrupciones
            como de excepciones, llaman a una función \textit{interrupt dispatcher}. Dicha función, en base al número de interrupción, se encarga de llamar
	    a la función que corresponda. Esto lo hace fijándose en una tabla, en la cual previamente fueron registradas las funciones que atienden
	    las interrupciones y excepciones soportadas por el sistema.
	    
	    En este punto, debemos hacer una mención a la int $80h$. Esta interrupción en particular, es una de las más importantes, ya que se encarga de manejar las
	    system calls. Es la encargada de conectar el \textit{kernel space} y el \textit{user space}. Es, por ejemplo, la que posibilita que funciones de la librería
	    estándar tengan acceso a funciones del kernel, como aquellas que interactúan con los perífericos. 
	   
	\subsubsection{Atendiendo excepciones}
	    En lo que a manejo de excepciones refiere, dado que nuestro sistema es mono tarea y que no implementamos nada que nos permita realizar algún tipo de acción 
	    reparadora, el nuestro es un manejo bastante simple. Nos limitamos a mostrar un mensaje de error, discriminando el tipo de excepción, y congelar el sistema,
	    deshabilitando las interrupciones.

    \subsection{System Calls}
	    Entre las system calls que implementamos se encuentran las que son básicas para el funcionamiento de nuestro sistema operativo, \textit{read} y \textit{write},
	    aunque también agregamos algunas más para facilitarnos el trabajo o, en algunos casos, para poder hacerlo bien. Estas system calls son \textit{ioctl}, 
	    \textit{time} y una system call creada especial creadas por nosotros llamada \textit{ticks}. A esta system call le asignamos un número de forma que no 
	    coincidiera con el número de ninguna otra system call existente.\footnote{El número en cuestión es 191.}
	
	\subsubsection{\textit{Kernel space} y \textit{user space}}
	    Si bien hemos dicho que existe un \textit{kernel space} y un \textit{user space}, y que la int $80h$ es la \textbf{única} conexión entre ellos, esto no es 
	    enteramente cierto. Dado que no tenemos control sobre los niveles de privilegio, porque no tenemos paginación ni segmentación, no podemos realmente fijar un
	    límite para ninguno de ellos. Lo máximo que pudimos hacer es fijar como pauta que las funciones que corren en \textit{user space} no puedan tener acceso a
	    las que están en \textit{kernel space}, a menos que utilicen una system call. Por supuesto, las funciones que corren en \textit{kernel space} no deberían 
	    llamar nunca a alguna función de \textit{user space}.

    \subsection{Terminales}
        \subsubsection{Entrada}
            La entrada se maneja mediante un buffer, que es actualizado cuando el \textit{PIC} dispara una interrupción de parte del controlador de teclado.
            Cuando esta interrupción se dispara, se interpreta los códigos enviados por el teclado y se agregan al buffer.
            La interpretación de estos códigos, difiere según el estado de ciertas teclas, como serlo Alt, Ctrl, Shift y Bloc Mayus.
            Luego, este buffer puede ser consultado utilizando la llamada \textit{read} sobre la entrada estándar.

            Como fue mencionado anteriormente, las terminales soportan varios modos de entrada.
            Por simplicidad, y por que proveen suficiente funcionalidad, se implementaron 4 modos en total.
            Qué modo se utiliza se puede manipular con la llamada \textit{ioctl}.

            El modo esta determinado por 2 flags: \textit{canon} y \textit{echo}.
            Canon activa y desactiva el modo Canónico de la terminal.
            Mientras que echo activa y desactiva si los caracteres leídos son impresos a pantalla por la terminal.

            El modo canónico determina un gran factor del comportamiento de llamar a \textit{read} sobre la entrada estándar.
            Si este esta activado, \textit{read} funciona como buffereado por línea, o sea, lee de a líneas.
            Esto significa que en una dada llamada, \textit{read} va a retornar como mucho una línea, y no va a retornar, hasta que tenga por lo menos una línea en su buffer.
            Si bien las terminales de linux permiten manipular que determina un fin de línea, nosotros tomamos \textit{\\n} como fin de línea.
            Si el modo canónico no esta activado, \textit{read} se comporta como su definición dice, lee n bytes, sin importar si hay líneas o no.

            Además de soportar estos 4 modos, el driver interpreta teclas especiales, como lo son las flechas.
            Algunas de estas teclas, como ser Bloc Mayus, Num Lock y Scroll Lock son manejadas internamente por el driver.
            Otras como las flechas, Inicio, Fin y las teclas de Función, se guardan como parte de la entrada previamente siendo convertidas a escape sequences.

            En particular, las teclas Block Mayus, Num Lock y Scroll Lock, modifican el estado de sus respectivos LEDs cuando son apretadas.
            Esto se hace enviando un comando al controlador de teclado mediante \textit{out}. 
            Al tomar tan solo 4 puertos de comunicación, el controlador de teclado tiene la particularidad de que es necesario introducir tiempos de espera.
            Esto es necesario para evitar escribir datos a un puerto donde el controlador esta sacando valores para la CPU, o viceversa.
            Es también curioso, que el cambio del estado de los LEDs genera que se dispare una interrupción de teclado con un valor de acknowledge.

            Por limitaciones del sistema, al no tener multi tarea, el buffer de teclado es uno solo, compartido por todas las terminales.

        \subsubsection{Salida}
            La comunicación entre el driver de salida de una terminal y la controladora de video se hace mediante el sector de video mapeado a memoria.
            Este permite suficiente flexibilidad para nuestros objetivos, y a su vez es muy simple de utilizar.
            La única excepción es la manipulación del cursor gráfico, que se hace utilizando \textit{in} y \textit{out}, ya que no encontramos si la posicion del mismo esta mapeada en alguna parte de memoria.

            El driver permite manipular todos sus aspectos mediante llamadas a \textit{write}.
            Este acepta escape sequences que dejan realizar un gran abanico de operaciones sobre la salida.
            El listado de las mismas esta en la documentación del driver de video.
            
            En este nivel se maneja un buffer de salida donde se guarda una copia del estado actual de la pantalla.
            Esto es así por 2 motivos.
            Primero, para evitar lecturas a la memoria de video cuando es necesario hacer scroll de pantalla.
            Segundo, es un elemento crucial a la hora de implementar múltiples terminales.
            Cada terminal tiene su propio estado en el driver de salida, incluyendo su propio buffer, a diferencia del driver de entrada.
            Cuando es pedido, el driver cambia cual de los estados almacenados usa, efectivamente cambiando de terminal.

    \subsection{Reboot} 
        Cuando llego la hora de implementar el reboot, encontramos que esto comúnmente se hace mediante el controlador de teclado.
        Este esta conectado al pin \textit{RESET} del microprocesador, y al recibir un comando especial, lo activa.
        Esto causa que la computadora haga un soft reboot, y dependiendo del controlador de teclado, es warm o cold.

\section{Librería estándar}
    Para poder realizar nuestro sistema operativo, fue necesario implementar cierta funcionalidad de la librería estándar de C. Al hacerlo, respetamos, siempre que fuese posible, las definiciones y las funcionalidades presentadas por el estándar \textit{ANSI}.
    Finalmente, una vez realizada esta librería, su funcionalidad podía ser aplicada en diferentes lugares del sistema operativo.

    Implementar la librería estándar de C de forma completa era innecesario para nuestros objetivos finales, por eso es que a la hora de hacerla, elegimos un subconjunto de la misma. El criterio para esta selección fue tomar las funciones que eran totalmente necesarias para nuestra implementación y además, otras que creiamos que cuya funcionalidad podría llegar a ser útil durante el desarrollo.
    Asimismo, el mismo criterio fue tomado para desarrollar la funcionalidad de funciones particulares. Véase como ejemplo la función \textit{printf}, que no permite hacer todo lo que permite hacer la librería estándar original (solo se pueden imprimir \textit{integers}, \textit{strings}, \textit{chars} y \textit{unsigned ints}).

    Por otra parte, en ciertos momentos nos vimos bajo la necesidad de implementar funciones no especificadas como parte del estándar. Un muy buen ejemplo era la función \textit{reverse} que da vuelta el contenido de un \textit{string}. En estos casos, se implementaron las funciones y fueron agregadas en los archivos correspondientes (en este caso en \textit{string.c}). Estas funciones fueron extensiones de nuestra librería estándar que fueron tomadas a gusto y bajo necesidad.

    Nuestra librería estándar incluye: funciones para el control del tipo de datos (\textit{isdigit}, \textit{isspace}, \textit{islower}, \textit{isupper}, \textit{isalpha} y \textit{isalnum}), funciones para entrada y salida de texto (\textit{fputc}, \textit{putc}, \textit{putchar}, \textit{puts}, \textit{fputs}, \textit{vfprinf}, \textit{vprintf}, \textit{fprintf}, \textit{printf}, \textit{fgetc}, \textit{getc}, \textit{getchar}, \textit{vfscanf}, \textit{vscanf}, \textit{fscanf} y \textit{scanf}), funciones para conversión de numeros a texto y viceversa (\textit{itoa}, \textit{atoi}, \textit{utoa} y \textit{atou}), generación de numeros al azar (\textit{rand} y \textit{srand}), manejo de cadenas de caracteres(\textit{strlen}, \textit{strcpy}, \textit{strncpy}, \textit{strcmp}, \textit{strncmp}, \textit{strchr}, \textit{strrchr}, \textit{strcat}, \textit{strncat}, \textit{reverse}), cierto manejo de sectores en memoria(\textit{memcpy}, \textit{memchr}, \textit{memset}, \textit{memcmp}), acceso simple al reloj para obtener el tiempo, manejo de argumentos variables y la definición de ciertas constantes necesarias.

    A continuación se presentará la especificación de ciertas funciones que creemos que vale la pena destacar ya sea por su funcionalidad o por problemas que nos encontramos a la hora de implementarlas:

    \subsection{Putc}
        Al comenzar, lo escencial era poder imprimir en pantalla. Por esto, una primer versión primitiva de \textit{putc} simplemente se encargaba de esto, llamando de forma directa a una función \textit{write} que imprimía en pantalla. De a poco \textit{putc} fue evolucionando y paso a hacer llamadas a sistema de forma correcta para imprimir como debía.
        Luego, comenzamos a tomar en cuenta que las funciones originales de la librería estándar, a la hora de imprimir, tomaban en cuenta en que archivo o flujo estaban imprimiendo. Al ver esto, tomamos la decisión de implementar una rudimentaria version de la estructura de datos \textit{FILE} con su file descriptor correcto. Así, también fueron definidas como estructuras de tipo \textit{FILE}, \textit{stdin}, \textit{stdout} y \textit{stderr} acorde al estándar. Dada la funcionalidad de nuestro trabajo, no tiene sentido alguno llamar funciones como \textit{fputc} con streams que no sean la salida estándar, pero se ha dejado la posiblidad para posibles expansiones de este trabajo.
        Por último cabe destacar que en versiones finales, funciones como \textit{putc} terminaron siendo simples macros que llaman a una función mas generica como \textit{fputc}.

    \subsection{Printf}
        Como ya fue mencionado, a la hora de implementar \textit{printf}, hubo que elegir un subconjunto de los tipos de datos que podía imprimir \textit{printf}. Implementarla de forma completa hubiese sido innecesario para nuestro trabajo. \textit{Printf} es un muy buen ejemplo de como, al realizar una función, es evidente la necesidad de otras funciones de la librería estándar aún no implementadas. Al momento de tener que imprimir un número entero, era intuitivo la necesidad de una función como \textit{itoa}, para poder imprimir un \textit{integer} de forma sencilla. Esta secuencia era natural al escribir la librería estándar y obviamente, este era otro muy buen criterio para decidir que funcionalidades implementar.

    \subsection{Scanf}    
        A la hora de implementar \textit{scanf}, primero hubo que tener cuidados similares a los que se tuvieron en \textit{printf}. Durante la implementación surgió la necesidad de devolver al flujo el último caracter consumido, lo cual naturalmente terminó siendo traducido en nuestra función \textit{ungetc}. \textit{Ungetc} es un ejemplo de como hubo que modificar cierto código viejo (en las estructuras de tipo FILE y en la función \textit{fputc}) a partir de funcionalidades necesarias sobre la marcha.
\newpage
\section{Shell}
    Contamos con una shell por cada terminal.
    Esta utiliza al máximo las capacidades de la terminal, comunicándose siempre mediante las funciones de la librería estándar, como \textit{printf}.
    La shell manipula el modo de la entrada de la terminal para permitirse manipular la entrada con gran precisión.
    Se desactiva tanto el \textit{echo} como el \textit{modo canónico}, o sea, lo ingresado llega solo a la shell y sin esperar a que se complete la línea.
    Esto le permite al ususario editar su comando a medida que lo escribe, auto completar y poder moverse por el historial de comandos.

    El comando \textit{sudoku} demuestra la capacidad de la terminal, haciendo uso de los colores y la posibilidad de mover el cursor a placer.
    
    La función del comando \textit{calc} es el de denotar el funcionamiento de \textit{scanf}, leyendo asi ciertos patrones.
    Es una calculadora muy rudimentaria y posee ciertos problemas respecto a como exhibe los datos.
    Sin embargo, la consideramos suficiente como para mostrar el funcionamiento de \textit{scanf}.
    
    Por último, el comando \textit{fortune}, ademas de ser un comando divertido de implementar, vuelve a mostrar otro uso de \textit{rand} de la librería estándar.

    Se mencionó antes que por la limitación de ser mono tarea, hay un solo buffer de teclado.
    Esto se nota también a nivel shell ya que solo se permite cambiar entre terminales cuando se esta ingresando un comando.

\newpage
\section{El comando GetCPUSpeed}
    Para la implementación del comando \textit{getCPUSpeed}, luego de investigar un poco, encontramos dos métodos posibles. Uno de ellos, el que a simple vista resultaba
    más simple de implementar, tenía la siguiente forma:
    \paragraph*{}
    \begin{lstlisting}
    setearTimer(X milisegundos);
    WHILE no se cumplio el tiempo:
	  incrementar contador;
    endwhile
    cpuFrecMHZ = (contador * ciclosClockPorIteracion) /  1000;
    \end{lstlisting}
    \paragraph*{}
    Veáse que en este pseudocódigo aparece ímplicitamente la "falla", o por lo menos "dificultad", de este método. Se debe saber, de alguna forma, la cantidad de ciclos
    de clock que toma una iteración. Esto no es nada trivial. En procesadores más "viejos", como los 286, 386 y 486, cada instrucción tardaba una cantidad fija y conocida
    de ciclos de clock en ejecutarse. Por lo que solo deberíamos buscar cuánto tarda cada instrucción que está dentro  del ciclo \textit{while} y sumarlas.

    Desafortunadamente, con la llegada de la tecnología \textit{pipeline}, esa cantidad de ciclos de clock para cada instrucción comenzó a ser variable, siendo la 
    razón más grande que el tiempo de una instrucción comenzó a depender de las instrucciones que la rodeaban.

    Debido a esto, y a pesar de que exiten formas de idear código cuyos ciclos de clock casi no se vean afectados por el \textit{pipeline}\footnote{ Con 
    subsiguientes instrucciones \textit{xor} que dependan de las instrucciones anteriores,por ejemplo alterando los registros que usan la instrucción siguiente.}, 
    decidimos que este método era "antiguo", ya que para "nuevos" procesadores existe una instrucción que nos facilita mucho el trabajo: \textit{rdtsc}.

    La instrucción \textit{rdtsc}, lee la cantidad de ciclos de clock desde el inicio del procesador del \textit{Time Stamp Counter}, un contador interno que viene
    en los procesadores a partir del Pentium. Existen formas de verificar la existencia de este contador interno pero asumimos que nuestro sistema correrá siempre en
    procesadores que soporten esta instrucción.

    Para realizar la medición de la frecuencia de trabajo de la \textit{CPU} con la instrucción \textit{rdtsc}, básicamente lo que tenemos que hacer es 
    \[  \text{frecuencia de trabajo de CPU} = \dfrac{\text{ciclos de clock}}{\text{tiempo}} \]
    
    Para hacer esto, nuestro algoritmo hace algo como esto:
    \paragraph*{}
    \begin{lstlisting}
    esperar a que se active el timmer tick;
    ticksAEsperar = n;
    i = 0;
    ciclosClockIni = rdtsc;
    WHILE i < n:
    endwhile
    ciclosClockFin = rdtsc;
    cpuFrecHZ = (ciclosClockFin - ciclosClockIni) / (ticks esperados * tiempo entre ticks);
    \end{lstlisting}
    \paragraph*{}   
    Este enfoque, resultó ser excesivamente preciso, teniendo en cuenta que nuestro sistema utiliza división entera.

    Dado que estas mediciones dependen del estado interno del procesador, y como no sabemos exactamente de qué y cómo dependen optamos por realizar varias mediciones
    y sacar un promedio para minimizar posibles errores. Por ejemplo, es de esperarse que en procesadores de varios núcleos las mediciones
    pierdan un poco de precisión, como mínimo.

\newpage     
\section{Referencias}

\begin{itemize}
  \item Material provisto por la cátedra
  \item The C programming language - Kernighan y Ritchie
  \item http://invisible-island.net/xterm/ctlseqs/ctlseqs.html
  \item http://webpages.charter.net/danrollins/techhelp/0087.HTM
  \item http://faydoc.tripod.com/cpu/rdtsc.htm
  \item http://stanislavs.org/helppc/
  \item http://www.linux.it/~rubini/docs/ksys/ksys.html
  \item http://wiki.osdev.org
  \item http://wiki.osdev.org/Detecting\_CPU\_Speed
  \item	http://wiki.osdev.org/CMOS\#Accessing\_CMOS\_Registers
  \item http://wiki.osdev.org/Bootable\_CD
  \item http://wiki.osdev.org/Boot\_sequence\#Easy\_Way\_Out
  \item http://en.wikipedia.org/wiki/System\_time\#Retrieving\_system\_time
  \item http://en.wikipedia.org/wiki/Calculating\_the\_day\_of\_the\_week
  \item http://cplusplus.com/

\end{itemize}
   
\end{document}
