\documentclass[a4paper,10pt]{article}

\usepackage[utf8]{inputenc}
\usepackage{t1enc}
\usepackage[spanish]{babel}
\usepackage[pdftex,usenames,dvipsnames]{color}
\usepackage[pdftex]{graphicx}
\usepackage{hyperref}
\usepackage{enumerate}
\usepackage{amsmath}
\usepackage{amsfonts}
\usepackage{amssymb}
\usepackage[table]{xcolor}
\usepackage[small,bf]{caption}
\usepackage{float}
\usepackage{subfig}
\usepackage{listings}
\usepackage{bm}
\usepackage{times}
\lstset{language=C}
\lstset{showstringspaces=false}
\lstset{basicstyle=\ttfamily,}

\begin{document}


\renewcommand{\lstlistingname}{C\'odigo Fuente}
\lstloadlanguages{Octave} 
\lstdefinelanguage{MyPseudoCode}[]{Octave}{
	deletekeywords={beta,det},
	morekeywords={repmat}
} 
\lstset{
	language=MyPseudoCode,
	stringstyle=\ttfamily,
	showstringspaces = false,
	basicstyle=\footnotesize\ttfamily,
	commentstyle=\color{gray},
	keywordstyle=\bfseries,
	numbers=left,
	numberstyle=\ttfamily\footnotesize,
	stepnumber=1,                   
	framexleftmargin=0.20cm,
	numbersep=0.37cm,              
	backgroundcolor=\color{white},
	showspaces=false,
	showtabs=false,
	frame=l,
	tabsize=4,
	captionpos=b,               
	breaklines=true,             
	breakatwhitespace=false,      
	mathescape=true
}
\begin{titlepage}
        \thispagestyle{empty}
        \begin{center}
                \includegraphics{./images/itba.jpg}
                \vfill
                \Huge{Sistemas Operativos}\\
                \vspace{1cm}
                \huge{Trabajo Práctico Especial}\\
        \end{center}
        \vspace{2cm}
        \large{
                \begin{tabular}{lcrc}
                        \textbf{Alvaro Crespo} & & 50758 & \ \ \texttt{acrespo@alu.itba.edu.ar}\\
                        \textbf{Juan Pablo Civile} & & 50453 & \ \ \texttt{jcivile@alu.itba.edu.ar}\\
                        \textbf{Darío Susnisky} & & 50592 & \ \ \texttt{dsusnisk@alu.itba.edu.ar}\\
                        \\ 
                \end{tabular}
        }
        \vfill
        \flushright{29 de Noviembre del 2011}
\end{titlepage}

\setcounter{page}{1}

\tableofcontents
\newpage
\section{Introducción}
El trabajo práctico consiste en agregarle funcionalidad y nuevas prestaciones al sistema operativo realizado en la materia 
Arquitectura de las Computadoras que luego fue extendido en esta misma materia. Esta nueva versión debía completar las falencias 
del trabajo anterior (incluyendo un conjunto completo de comandos para poder manipular el \textit{file system} a gusto del 
usuario), una unidad de paginación, \textit{stacks} y \textit{heaps} protegidos y privados para cada proceso (como desafío adicional 
el stack debía ser dinámico) y la implementación de una memoria \textit{caché} de disco que posea métodos adecuados para su correcto 
funcionamiento.

\newpage
\section{Breve resumen de la vieja versión de Arqvengers OS}
El trabajo práctico fue comenzado tomando como base el trabajo hecho en la materia Arquitectura de las Computadoras y la extensión 
realizada al mismo en esta materia.

La primer versión del trabajo consistía en un sistema operativo booteable que contaba con soporte para \textit{drivers} de
 teclado y de video facilitando varias consolas en donde podían ejecutarse distintos comandos. 
A un nivel más bajo, contábamos con una rudimentaria e incompleta librería de C, así como una interfaz para realizar 
llamadas a sistema.

En nuestra segunda versión se mejoró el manejo de los procesos agregando funcionalidad multitarea con dos técnicas de 
\textit{scheduling} diferentes. Se agregó soporte para distintos usuarios y un \textit{driver ATA} para poder contar con persistencia 
de la información. Además se implementó un \textit{file system} que entre otras funcionalidades manejaba FIFOs, permisos, 
archivos ocultos, \textit{links} simbólicos y directorios. En compañía de esto, nos vimos obligados a extender nuestra librería de 
C y de proveerle al usuarios un conjunto de comandos para poder manipular el \textit{file system} y poder probar las nuevas 
prestaciones de nuestro sistema operativo.

\newpage

\section{Agregados y correcciones correspondientes al trabajo previo}

Además de corregir ciertos comportamientos inesperados (había ciertos problemas en el comando \textit{pwd} y con los
\textit{links} simbólicos), era correspondiente agregar a los comandos ya presentes \textit{mv} y \textit{cp}. Este último, 
debía aceptar la opción ``-r'' para realizar una copia recursiva entre directorios. La forma en la que estaba implementado 
el \textit{file system} (contábamos con soporte Ext2) y los funcionalidades ya existentes hicieron de esta una tarea más que sencilla.

\newpage

\section{Unidad de paginación}

\subsection{Manejo de page faults}

Trás detectar que se estaba generando una excepción de tipo \textit{page fault}, era necesario implementar un manejador de estas 
excepciones. Dados los requerimientos del trabajo práctico, solamente era necesario tomar acciones especiales en el caso de que 
se necesite más espacio en el \textit{stack} (detallado más adelante). Por suerte, se contaban con los datos necesarios para detectar en que 
caso estaba sucediendo esto. Dentro de los registros de uso común del micro, se contaban con código de error y en el registro \textit{CR2} 
se encontraba la dirección lógica que generó el \textit{page fault}.

En los casos en que no se puede manejar la excepción o se genera algun tipo de problema al resolverlo, se procede terminando la 
ejecución del proceso que generó dicha excepción.

\newpage

\section{Stack y heap de un proceso}

Otra de las funcionalidades agregadas en este trabajo es que cada proceso cuenta con un sector en memoria para su \textit{stack} y 
otro para su \textit{heap}. Estos sectores de memoria debían ser exclusivos para el proceso y accesibles únicamente por él.

La unidad de paginación presente en nuestro sistema operativo posibilita reservar páginas para cada proceso, por lo que si otro 
proceso pretende acceder a esta página se genera un \textit{page fault}. Asimismo, nuestro sistema facilita que las 
páginas físicas reservadas para un proceso sean mapeadas en sectores constantes en el directorio de páginas. De este modo, 
todos los procesos tienen su \textit{stack} comenzando siempre en los 3GB (lógicos) pero la únidad de paginación permite que las 
páginas de cada proceso sean diferentes.

\subsection{Stack dinámico}

Al comenzar un proceso se reservan una cantidad de páginas para el \textit{stack} del mismo. Es posible que durante la ejecución, 
este tamaño inicial no sea suficiente. En este caso, se tratará de acceder a la página contigua al tope del \textit{stack} derivando 
en un \textit{page fault}. Dentro del manejador de \textit{page faults} es posible verificar estas últimas condiciones. En este caso, 
es necesario reservar una nueva página para el proceso y mapearla en la dirección que disparó el \textit{page fault} (contenida en 
el registro \textit{CR2}). 

Por último, es importante decidir un tamaño máximo para el \textit{stack} de un proceso para que no se abuse de las páginas 
reservadas.

\newpage

\section{Memoria caché}

Nuestro sistema operativo cuenta con una capa de caché entre \textit{ext2} y el driver \textit{ATA}. Para implementar esta capa decidimos usar una tabla 
a nodos de una lista doblemente linkeada (\textit{Doubly-Linked List}), ordenada por el criterio \textit{LRU} (\textit{Least Recently Used}), el menos usado
 recientemente. Cada nodo es un pedazo (\textit{chunk}) de memoria con la información inherente a él (sector inicial, \textit{flag de dirty}, 
cantidad de accesos, última vez que fue accedido, etc...). Esta estructura de datos resulta vital para las tareas a realizar ya que permite tener una lista
en orden constante y de acceso constante.\\

Después de sobre el asunto, decidimos que estos pedazos de memoria tengan tamaño (cantidad de sectores)
fijo y fueran simplemente páginas. Esta decisión nos facilitó el posterior trabajo.\\

El comportamiento de la caché es el típico de esta tipo de memoria. Cada vez que se lee o escribe al disco, si la información no se encuentra en la caché, se
 la almacena para accesos futuros. Luego, si se quiere volver a acceder a dicha información, estará disponible en memoria, sin necesisdad de acceder al disco, lo 
cual es una operación costosa.\\


Nuestra tabla tiene 256 entradas por lo que nuestra caché tiene un tamaño total de 1 Mb (256 páginas x 4 Kb).\\

        \subsection{Técnicas de sincronización}

        Cuando el sistema escribe información en la caché, en algún momento debe actualizar esa información en el disco. Una parte importante de la caché es cuándo y
        cómo se sincroniza esa información que se tiene en memoria con su contraparte en disco. \\

        Existe 2 técnicas básicas de llevar esto a cabo:

        \begin{itemize}
        \item \textit{write-through} (escribir a través), cada vez que se escribe a caché se escribe también a disco.
        \item \textit{write-back} (escribir por detrás), la escritura se hace solo a caché, luego, la información modificada es escrita a disco en algún momento posterior.
        \end{itemize}

        Por supuesto, la implementar la técnica de \textit{write-back} es mucho más complejo, dado que se necesita mantener un registro de que sectores han sido escritos y 
        marcarlos como \textit{dirty} (``sucios'') para que sean luego escritos a disco. La información en estos sectores es escrita a disco sólo cuando son desalojados
        de la caché. (cache eviction). \\

        Inicialmente, para probar nuestra capa de caché, implementamos la técnica de \textit{write-through}, para probar el funcionamiento de la misma. 
        Luego, extendimos su funcionalidad para implementar \textit{write-back}. \\

        El criterio que elegimos para sincronizar una página cacheada fue bastante sencillo: si una página lleva más de un segundo ``sucia'', se debe escribir a disco.
        %TODO subsubsection???
        Para controlar esta tarea, decidimos tener un proceso especial que la lleve a cabo. Este proceso simplemente espera un cantidad determinada de tiempo y realiza 
        la sincronización, y el desalojo de ser necesario, de las páginas de la memoria caché.


\newpage
\section{Agregados}

        \subsection{Comandos para demostrar funcionalidad del sistema}
        %TODO esta bien esto aca? o una subsection para cada comando?
        
        Para demostrar el funcionamiento de la unidad de paginación y la protección de memoria implementamos un simple comando \textit{rma} (Random Access Memory), 
        que acceda a un sector de memoria elegido de forma aleatoria. De esta forma se puede observar como el sistema se da cuenta que (en la gran mayoría de 
        los casos) se está intentando acceder a un sector de memoria que es ajeno al proceso que esta corriendo y no lo permitirá. Para ver más acerca de este 
        comando se puede ver su código fuente, en el archivo /src/shell/rma/rma.c.\\

        Para demostrar el crecimiento en forma dinámica del \textit{stack} de un proceso implementamos el comando \textit{stacksize}, el cual hace crecer el 
        \textit{stack} del proceso hasta un tamaño determinado (el cual recibe como argumento) o de forma indefinida (si no se le proporciona un tamaño tope).
        Se puede observar en la consola como el comando corre un tiempo hasta que termine abruptamente su ejecución cuando alcanza el tamaño máximo determinado 
        o cuando sobrepasa la máxima capacidad para el \textit{stack} de un proceso. Dicha capacidad esta fijada en 1024 páginas, es decir $4 Mb$.\\

        Además, puede observarse en la \textit{Log Terminal} como el sistema va reservando páginas de forma incremental.

        \subsection{Log Terminal}

        Decidimos habilitar una quinta terminal reservada exclusivamente usos de \textit{logging}. 
        Estos es, el sistema puede dejar registro de ciertas acciones que realiza o errores que ocurran y esta terminal permite al usuario ver este registro. De más
        está decir que está terminal es de sólo lectura, el usuario no puede escribir o realizar cambios.\\

        Para esto fue necesario implementar una serie funciones y tomar ciertos recaudos pero, finalmente creemos que el resultado vale la pena.

        Decidimos implementar 4 níveles de \textit{logging} y un comando \textit{loglevel} para que el usuario puede elegir el que desee.\\

        \begin{itemize}
        \item QUIET, la terminal no mostrará ningún tipo de registro.
        \item ERROR, se registrarán solo errores que ocurran en el sistema.
        \item INFO, se registrará además información sobre las acciones del sistema.
        \item DEBUG, se registrará también datos sobre depuraciones del sistema.
        \end{itemize}

        La decisión de implementar esta \textit{Log Terminal}, resultó muy útil tanto a la hora de testear y depurar nuestro sistema, como a la hora de demostrar al 
        usuario las distintas funcionalidades del mismo (protección de memoria, cache de disco, stack creciente de forma dinámica, etc...).\\


        \subsection{Memory managment}
        Inicialmente, el trabajo práctico contaba con múltiples arrays de tamaño estático para guardar todos los datos internos del kernel.
        Por supuesto, esta solución es menos que óptima, ya que genera un desperdicio de memoria e impone límites arbitrarios al sistema.
        Para evitar esto, desarrollamos un sistema de allocación de páginas de memoria.
        Y encima de este sistema, utilizamos la implementación de \textit{malloc} de Dario Sneidermanis, publicada en \href{https://github.com/esneider/malloc}{Github}.

        Para allocar páginas, hicimos uso de información provista por \textit{Grub} que permite saber exactamente qué zonas de memoria se pueden utilizar, 
        y cuanta memoria hay disponible.
        Esta memoria la dividimos en bloques de $4KB$, teniendo en mente la posibilidad de agregar paginación al sistema.
        Para obtener estas páginas optamos por un simple algoritmo de allocación basado en un bitmap de uso de páginas.

        Una característica interesante de la implementación de \textit{malloc}, es que permite tener varios contextos de memoria distintos.
        Lo que facilitó tener un contexto de memoria para el kernel, y otro para cada proceso (excepto los procesos de kernel que usan el mismo que el kernel).
        La única limitación del sistema es que los contextos de proceso tienen un tamaño fijo, mientras que el del kernel crece a medida que es necesario.\\


        \subsection{Terminal control sequences}
        Nuestra implementación de terminal soporta una gran parte de \href{http://invisible-island.net/xterm/ctlseqs/ctlseqs.html}{xterm control sequences}.
        Estas permiten manipular el cursor y el color de la pantalla desde los procesos utilizando sólo la primitiva \textit{write}.


\newpage
\section{Problemas encontrados}

Durante el desarrollo de este trabajo práctico fueron surgiendo diferentes dudas y problemas. El propósito de esta
sección es comentarlos con el fin reflejar proceso de aprendizaje.

Un dilema bastante frecuente a lo largo del trabajo práctico fue lo que nos gusta llamar el problema de ``la gallina o el huevo''. Al realizar
un desarrollo a bajo nivel que ha comenzado con pocos recursos, era habitual encontrarse con la duda de si convenía
arrancar a implementar la lectura o la escritura de alguna prestación. Es evidente que si se hacía primero la lectura, 
no había forma de probar su funcionamiento ya que no existía la escritura. Lo mismo sucedía a modo inverso. Si bien 
la solución no era complicada (desarrollar todo) era incómodo y molesto a la hora de encontrar errores ya que los mismos
podían encontrarse en cualquier lado.

Varios problemas interesantes surgieron durante la creación del \textit{driver} ATA. A pesar de que se siguieron
rigurosamente los estándares, se encontró que no todos los emuladores funcionaban de forma correcta con el mismo.
Inclusive, detectamos que el driver no se comportaba de manera deseada al tratar con varios sectores a la vez con lo
cual nos vimos forzados a modificar la lógica interna del driver simulando un acceso a múltiples sectores a tráves de
varios accesos de a un sector a la vez.


        %TODO WTF? Que hacemo con estas subsecciones??En esta section no cuadran a donde las movemos???
        \subsection{Comandos de shell}
        Durante el desarrollo del file system, eventualmente llegamos al punto en el que necesitábamos implementar \textit{cd}.
        Una vez terminaba la implementación, notamos que no funcionaba, pero todo el output de debugging indicaba que si.
        Finalmente, nos dimos cuenta de que \textit{cd} ejecutaba en un proceso distinto al de la shell.
        Entonces cambiaba su propio directorio, pero no el de la shell.
        Para solucionar este problema necesitamos introducir el concepto de comandos internos de la shell.
        Véase, comandos que se ejecutan dentro del mismo proceso que la shell.

        \subsection{Scrolling}
        Dado el alto número de comandos que presenta el ArqvengerOS, llegó un momento en el que la lista de todos ellos no entraba en una sola pantalla.
        Como implementar scrolling vertical en la terminal presenta un desafio importante, optamos por una solución más simple pero igual de efectiva.
        Los comandos que presentan grandes cantidades de información, como \textit{help} y \textit{ls}, cuando van a escribir más alla de altura de la pantalla, 
        paginan la salida.
        El comportamiendo es muy parecido al de la utilidad \textit{more}, pero se implementó en casos particulares y no como una utilidad genérica. \\


        Como funcionalidad extra, y dado que la forma en que implementamos las cosas nos facilitó en gran parte la tarea, ofrecemos la posibilidad de scrollear una 
        pantalla para arriba o para abajo en todo momento usando las teclas \textit{RePag} y \textit{AvPag} respectivamente.


        \subsection{Kernel-space y user-space}
        Inicialmente nuestro objetivo era contar con una delimitación fuerte de que es \textit{kernel-space} y \textit{user-space}.
        Esto es, que el kernel ejecute en ring 0 y los procesos en ring 3.
        Pero esto resultó en una serie de excepciones del procesador.
        Hacer los cambios de segmento entre segmentos de distintos privilegios sin ayuda de algún mecanismo especial no parece ser posible.

        Idealmente, esto se solucionaría haciendo uso de Task Gates.
        Que también permitiría tener un stack dedicado para el kernel, para uso dentro de los manejadores de interrupciones.
        Si bien esta es la mejor solución disponible, no es trivial de implementar y sus beneficios están fuera del enfoque del trabajo práctico.

        \subsection{Cálculo de porcentaje de CPU utilizado}
        Para completar la implementación del comando \textit{top}, se requería calcular el porcentaje de CPU utilizado por cada proceso, en un determinado
        período de tiempo. Esta tarea no fue sencilla, por varias razones. Primero, se debe determinar que es lo que se medirá, tiempo, ticks (IRQ8) o ciclos. 
        Segundo, se debe determinar cuándo se tomarám las mediciones o actualizarán los valores. 

        En algunos primeros intentos se pensó, erronéamente, en contar ticks, o tomar las mediciones cada vez que llegara un interrupción de \textit{timmer tick} (IRQ8).
        Pero todas esas propuestas tenían alguna falla de algún tipo: o eran aproximaciones, o sencillamente no respondían al objetivo buscado.

        Finalmente, se decidió que claramente la mejor opción era contar ciclos clock (\textit{Clock Cycles}) y tomar las mediciones en cada cambio de contexto.
        De esta forma, obtenemos un medición muchísimo más precisa, y la complejidad de implementar esta tarea no fue tan grande.

\newpage     
\section{Referencias}

Para poder realizar este trabajo, se llevo a cabo un gran proceso de investigación y aprendizaje. Estas son algunas fuentes que resultaron útiles para realizar 
este trabajo.\\

\begin{itemize}
  \item Material provisto por la cátedra
  \item The C programming language - Kernighan y Ritchie
  \item http://invisible-island.net/xterm/ctlseqs/ctlseqs.html
  \item http://webpages.charter.net/danrollins/techhelp/0087.HTM
  \item http://faydoc.tripod.com/cpu/rdtsc.htm
  \item http://stanislavs.org/helppc/
  \item http://www.linux.it/~rubini/docs/ksys/ksys.html
  \item http://wiki.osdev.org
  \item http://wiki.osdev.org/Detecting\_CPU\_Speed
  \item	http://wiki.osdev.org/CMOS\#Accessing\_CMOS\_Registers
  \item http://wiki.osdev.org/Bootable\_CD
  \item http://wiki.osdev.org/Boot\_sequence\#Easy\_Way\_Out
  \item http://wiki.osdev.org/Ext2
  \item http://wiki.osdev.org/IDE
  \item http://wiki.osdev.org/ATA\_PIO\_Mode
  \item http://en.wikipedia.org/wiki/System\_time\#Retrieving\_system\_time
  \item http://en.wikipedia.org/wiki/Calculating\_the\_day\_of\_the\_week
  \item http://cplusplus.com/
  \item http://github.com/esneider/malloc
  \item Unix System Programming, Second Edition - K. Haviland, D. Gray, B. Salama.
  \item http://www.nongnu.org/ext2-doc/ext2.html
  \item http://wiki.osdev.org/Blocking\_Process
  \item http://forum.osdev.org/viewtopic.php?f=1\&t=21719
  \item http://f.osdev.org/viewtopic.php?f=1\&t=18738
  \item http://wiki.osdev.org/Context\_Switching
  \item http://wiki.osdev.org/Scheduling\_Algorithms
  \item http://en.wikipedia.org/wiki/Scheduling\_(computing)
  \item http://en.wikipedia.org/wiki/Round-robin\_scheduling
  \item http://en.wikipedia.org/wiki/Nice\_(Unix)
  \item http://en.wikipedia.org/wiki/Htop
  \item http://en.wikipedia.org/wiki/Top\_(Unix)
  \item http://www.suse.de/~agruen/acl/linux-acls/online/
  \item http://www.gnu.org/s/hello/manual/libc/Users-and-Groups.html
  \item http://www.gnu.org/s/hello/manual/libc/User-and-Group-IDs.html\#User-and-Group-IDs
  \item http://www.gnu.org/s/hello/manual/libc/Process-Persona.html\#Process-Persona
  \item http://www.gnu.org/s/hello/manual/libc/User-Database.html\#User-Database
  \item http://www.gnu.org/s/hello/manual/libc/Group-Database.html\#Group-Database
%Nuevos para este tp3
  \item http://en.wikipedia.org/wiki/Cache
  \item http://en.wikipedia.org/wiki/Cache\_algorithm
 
\end{itemize}
   
\end{document}
