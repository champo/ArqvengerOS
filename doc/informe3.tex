\documentclass[a4paper,10pt]{article}

\usepackage[utf8]{inputenc}
\usepackage{t1enc}
\usepackage[spanish]{babel}
\usepackage[pdftex,usenames,dvipsnames]{color}
\usepackage[pdftex]{graphicx}
\usepackage{hyperref}
\usepackage{enumerate}
\usepackage{amsmath}
\usepackage{amsfonts}
\usepackage{amssymb}
\usepackage[table]{xcolor}
\usepackage[small,bf]{caption}
\usepackage{float}
\usepackage{subfig}
\usepackage{listings}
\usepackage{bm}
\usepackage{times}
\lstset{language=C}
\lstset{showstringspaces=false}
\lstset{basicstyle=\ttfamily,}

\begin{document}


\renewcommand{\lstlistingname}{C\'odigo Fuente}
\lstloadlanguages{Octave} 
\lstdefinelanguage{MyPseudoCode}[]{Octave}{
	deletekeywords={beta,det},
	morekeywords={repmat}
} 
\lstset{
	language=MyPseudoCode,
	stringstyle=\ttfamily,
	showstringspaces = false,
	basicstyle=\footnotesize\ttfamily,
	commentstyle=\color{gray},
	keywordstyle=\bfseries,
	numbers=left,
	numberstyle=\ttfamily\footnotesize,
	stepnumber=1,                   
	framexleftmargin=0.20cm,
	numbersep=0.37cm,              
	backgroundcolor=\color{white},
	showspaces=false,
	showtabs=false,
	frame=l,
	tabsize=4,
	captionpos=b,               
	breaklines=true,             
	breakatwhitespace=false,      
	mathescape=true
}
\begin{titlepage}
        \thispagestyle{empty}
        \begin{center}
                \includegraphics{./images/itba.jpg}
                \vfill
                \Huge{Sistemas Operativos}\\
                \vspace{1cm}
                \huge{Trabajo Práctico Especial}\\
        \end{center}
        \vspace{2cm}
        \large{
                \begin{tabular}{lcrc}
                        \textbf{Alvaro Crespo} & & 50758 & \ \ \texttt{acrespo@alu.itba.edu.ar}\\
                        \textbf{Juan Pablo Civile} & & 50453 & \ \ \texttt{jcivile@alu.itba.edu.ar}\\
                        \textbf{Darío Susnisky} & & 50592 & \ \ \texttt{dsusnisk@alu.itba.edu.ar}\\
                        \\ 
                \end{tabular}
        }
        \vfill
        \flushright{29 de Noviembre del 2011}
\end{titlepage}

\setcounter{page}{1}

\tableofcontents
\newpage
\section{Introducción}
El trabajo práctico consiste en agregarle funcionalidad y nuevas prestaciones al sistema operativo realizado en la materia 
Arquitectura de las Computadoras que luego fue extendido en esta misma materia. Esta nueva versión debía completar las falencias 
del trabajo anterior (incluyendo un conjunto completo de comandos para poder manipular el \textit{file system} a gusto del 
usuario), una unidad de paginación, \textit{stacks} y \textit{heaps} protegidos y privados para cada proceso (como desafío adicional 
el stack debía ser dinámico) y la implementación de una memoria \textit{caché} de disco que posea métodos adecuados para su correcto 
funcionamiento.

\newpage
\section{Breve resumen de la vieja versión de Arqvengers OS}
El trabajo práctico fue comenzado tomando como base el trabajo hecho en la materia Arquitectura de las Computadoras y la extensión 
realizada al mismo en esta materia.

La primer versión del trabajo consistía en un sistema operativo booteable que contaba con soporte para \textit{drivers} de
 teclado y de video facilitando varias consolas en donde podían ejecutarse distintos comandos. 
A un nivel más bajo, contábamos con una rudimentaria e incompleta librería de C, así como una interfaz para realizar 
llamadas a sistema.

En nuestra segunda versión se mejoró el manejo de los procesos agregando funcionalidad multitarea con dos técnicas de 
\textit{scheduling} diferentes. Se agregó soporte para distintos usuarios y un \textit{driver ATA} para poder contar con persistencia 
de información. Además se implementó un \textit{file system} que entre otras funcionalidades manejaba FIFOs, permisos, 
archivos ocultos, \textit{links} simbólicos y directorios. En compañía de esto, nos vimos obligados a extender nuestra librería de 
C y de proveerle al usuarios un conjunto de comandos para poder manipular el \textit{file system} y poder probar las nuevas 
prestaciones de nuestro sistema operativo. 

Además el sistema ya contaba con un manejo de memoria bastante avanzado, con un \textit{heap} individual por proceso y una implementación
funcional de \textit{malloc}.

Como hay requisitos de este trabajo práctico que fueron desarrollados en versiones anteriores, no son incluidos en este informe ya que fueron
objeto de informes anteriores. Ante cualquier duda sobre estos temas referirse a ellos y/o contactarnos.

\newpage

\section{Agregados y correcciones al trabajo previo}

Además de corregir ciertos comportamientos inesperados (había ciertos problemas en el comando \textit{passwd} y con los
\textit{links} simbólicos), era correspondiente agregar a los comandos ya presentes \textit{mv} y \textit{cp}. Este último, 
debía aceptar la opción ``-r'' para realizar una copia recursiva entre directorios. La forma en la que estaba implementado 
el \textit{file system} (contábamos con soporte Ext2) y las funcionalidades ya existentes hicieron de esta una tarea más que sencilla.

\newpage

\section{Proceso}
\label{sec:Proceso}

En este trabajo práctico hicimos una cantidad significativa de cambios a la estructura de proceso, y al manejo de los mismos a nivel permisos.
Principalmente mejoramos el manejo de la memoria asignada a cada proceso, y implementamos un verdadero \textit{User-space}.

\subsection{User-space}
Una de las principales deficiencias de versiones anteriores de este sistema operativo era la ausencia de un verdadero \textit{User-space}.
Es decir, todos los procesos ejecutaban en $ring 0$ con el máximo nivel de permisos, y con acceso a toda la memoria.
En esta entrega, esta deficiencia no existe más.
Esto lo logramos en 2 partes: ejecución en $ring 3$ y paginación.
En esta seccion nos vamos a concentrar en las ventajas de ejecutar en $ring 3$. 
Paginación tiene su propia sección más adelante.(Ver sección \ref{sec:Paginación})


La transición de $ring 0$ a $ring 3$ es la piedra fundamental de la construcción de muchos aspectos de esta nueva entrega.
Esta nos permite hacer uso de features de \textit{x86} como lo es el \textit{Task State Segment}.
Si bien nuestro sistema operativo hace \textit{Software Task Switching} (Cambio de Tarea por Software), es posible utilizar un \textit{TSS} sin hacer uso de 
\textit{Hardware Task Switching}, (Cambio de Tarea por Hardware).
Alcanza con definir un solo \textit{Task State} y decirle al procesador que este es la tarea actual 
(que no hace ninguna clase de cambio al estado actual de ejecución).

Cabe recordar que existen 2 tipos de procesos: procesos de kernel, y procesos de usuario.
Los procesos de usuarios son los que ejecutan en $ring 3$, mientras que los procesos de kernel continúan disfrutando de ejecutar en el mismo contexto 
que el \textit{kernel} en sí. O sea, en $ring 0$ y con acceso a toda la memoria.

\subsection{Kernel stack}

Cada \textit{Task State} puede definir un \textit{esp0} que es el \textit{stack} que el microprocesador debe usar cuando se cambia de $ring 3$ a $ring 0$.
Es decir, cuando ocurre cualquier tipo de acción que provoca un cambio de nivel de privilegio (por ejemplo una interrupción), automáticamente el microprocesador 
comienza a utilizar el \textit{stack} ahí definido.
Esta funcionalidad nos da una posibilidad muy interesante: tener un \textit{stack} para manejo de interrupciones para cada proceso.

Este \textit{stack}, conocido como \textit{kernel stack} es único para cada proceso, y todo proceso tiene uno.
Ahora, cabe destacar que, por las características de los procesos de kernel, estos no poseen más que un \textit{kernel stack} y no disponen de un \textit{stack} regular.
El \textit{kernel stack} es de tamaño fijo, con tan solo $ 4KB $ de memoria a su disposicion.
Esta cantidad de memoria es más que suficiente para todas las acciones del \textit{kernel}.

Entre las ventajas de esto se pueden resaltar la indepencia entre el estado de un proceso de usuario, y el manejo de sus interrupciones.
Y por sobre todo, le da la capacidad al \textit{kernel} de ser \textit{reentrant}.
Es decir, puede abandonar la ejecución de una interrupción causada por un proceso a mitad de camino, y continuar atendiendo a otro proceso.
Esto nos permite bloquear a un proceso dentro de una \textit{system call} (como por ejemplo read y write), y pasar la ejecución a otro proceso sin que el primero 
se entere de lo sucedido.

\newpage
\section{Memory Managment Unit}

\subsection{Paginación}
\label{sec:Paginación}
Como parte de los objetivos de esta entrega, se encuentra habilitar el uso de paginación.
Para implementar esto, comenzamos habilitando una tabla de páginas muy simple, que asigna a todas las direcciones lógicas, la misma direccion física.
Una vez que logramos que esto funcione, pasamos a tener una tabla de páginas por proceso de usuario, y otra para el \textit{kernel} y procesos de kernel.

La tabla de \textit{kernel} es la misma que utilizamos en la primer implementación mencionada.
Esto permite tener acceso a toda la memoria, cosa esencial para su correcto manejo.

Cada proceso de usuario tiene su propia tabla.
En esta sólo esta visible la memoria que es estrictamente necesaria para la ejecución del proceso.
O sea, su código, sus \textit{stacks}, su \textit{heap} y una sección especial de variables globales (mejor explicado en detalle en la sección  
\ref{sec:Problemas encontrados}).

El cambio de tablas ocurre al principio y al final del punto de entrada de las interrupciones.
Cuando comienzan a ejecutar se coloca la tabla de \textit{kernel} y cuando termina se coloca la del proceso a ejecutar.

\subsection{Virtual addressing}

Al habilitar paginación, una de las posbilidades que se tiene es utilizar \textit{Virtual Addresing}.
Nosotros hacemos uso de esta feature para el \textit{stack} de ejecución de procesos de usuario.
Este \textit{stack} siempre está dispuesto de forma tal que el proceso acceda a él entre los $ 2GB $ y $ 3GB $.
La primer página esta ubicada inmediatamente antes de la dirección de $ 3GB $.

\subsection{Stack dinámico}

Al comenzar un proceso se reservan una cantidad de páginas para el \textit{stack}.
Es posible que durante la ejecución, este tamaño inicial no sea suficiente.
Por este motivo, cuando un proceso de usuario se queda sin espacio en su \textit{stack}, este es automáticamente extendido.
Esto ocurre de manera transparente al proceso haciendo uso de \textit{Page Faults}.

Para poder extenderlo, sin necesitar de páginas de memoria contiguas, es que decidimos colocar el \textit{stack} en una dirección de memoria lógica fija.
Haciendo uso de \textit{Virtual addressing} podemos simplemente colocar una página en la tabla del proceso, retornar de la interrupción, y continuar ejecutando
 como si nada hubiera pasado.

Por supuesto, cada proceso tiene un límite al tamaño del \textit{stack}.
En caso de que un proceso intente utilizar más memoria de la permitida, se lo termina.

\subsection{Manejo de page faults}

Trás detectar que se estaba generando una excepción de tipo \textit{Page Fault}, era necesario implementar un manejador de estas excepciones.
Dados los requerimientos del trabajo práctico, solamente era necesario tomar acciones especiales en el caso de que se necesite más espacio en el \textit{stack}.
Por suerte, se contaban con los datos necesarios para detectar en que caso estaba sucediendo esto.
El micro provee a la hora de un \textit{Page Fault} un codigo de error que indica que tipo de fault ocurrió, y en el registro \textit{CR2} se encuentra la dirección 
lógica que lo generó.

En los casos en que no se puede manejar la excepción o se genera algún tipo de problema al resolverlo, se procede terminando la ejecución del proceso que 
generó dicha excepción.

\newpage

\section{Memoria caché}
\label{sec:Memoria caché}
Nuestro sistema operativo cuenta con una capa de caché entre \textit{ext2} y el driver \textit{ATA}. Para implementar esta capa decidimos usar una tabla 
a nodos de una lista doblemente linkeada (\textit{Doubly-Linked List}), ordenada por el criterio \textit{LRU} (\textit{Least Recently Used}), el menos usado
 recientemente. Cada nodo es un pedazo (\textit{chunk}) de memoria con la información inherente a él (sector inicial, \textit{flag de dirty}, 
cantidad de accesos, última vez que fue accedido, etc...). Esta estructura de datos resulta vital para las tareas a realizar ya que permite tener una lista
en orden constante y de acceso constante.\\

Después de meditar sobre el asunto, decidimos que estos pedazos de memoria tengan tamaño (cantidad de sectores)
fijo y fueran simplemente páginas. Esta decisión nos facilitó el posterior trabajo.\\

El comportamiento de la caché es el típico de comportamiento este tipo de memoria. Cada vez que se lee o escribe al disco, si la información no se encuentra en la caché, se
 la almacena para accesos futuros. Luego, si se quiere volver a acceder a dicha información, estará disponible en memoria, sin necesisdad de acceder al disco, lo 
cual es una operación costosa.\\


Nuestra tabla tiene 256 entradas por lo que nuestra caché tiene un tamaño total de $1 MB$ (256 páginas x $4 KB$).\\

        \subsection{Técnicas de sincronización}

        Cuando el sistema escribe información en la caché, en algún momento debe actualizar esa información en el disco. Una parte importante de la caché es cuándo y
        cómo se sincroniza esa información que se tiene en memoria con su contraparte en disco. \\

        Existe 2 técnicas básicas de llevar esto a cabo:

        \begin{itemize}
        \item \textit{write-through} (escribir a través), cada vez que se escribe a caché se escribe también a disco.
        \item \textit{write-back} (escribir por detrás), la escritura se hace solo a caché, luego, la información modificada es escrita a disco en algún momento posterior.
        \end{itemize}

        Por supuesto, implementar la técnica de \textit{write-back} es mucho más complejo, dado que se necesita mantener un registro de que sectores han sido escritos y 
        marcarlos como \textit{dirty} (``sucios'') para que sean luego escritos a disco. La información en estos sectores es escrita a disco en algún momento posterior
        según criterio de sincronización implementado o cuando son desalojados de la caché (\textit{cache eviction}). \\

        Inicialmente, implementamos la técnica de \textit{write-through},  para probar el funcionamiento de nuestra capa de caché, 
        Luego, extendimos su funcionalidad para implementar \textit{write-back}. \\

        El criterio que elegimos para sincronizar una página cacheada fue bastante sencillo: si una página lleva más de un segundo ``sucia'', se debe escribir a disco.
        
        Para controlar esta tarea, decidimos tener un proceso especial que la lleve a cabo. Este proceso simplemente espera un cantidad determinada de tiempo y realiza 
        la sincronización, y el desalojo de ser necesario, de las páginas de la memoria caché. 

        \subsection{Cache eviction}
        

\newpage
\section{Agregados}

        \subsection{Comandos para demostrar funcionalidad del sistema}
             
        Para demostrar el funcionamiento de la unidad de paginación y la protección de memoria implementamos un simple comando \textit{rma} (Random Access Memory), 
        que accede a un sector de memoria elegido de forma aleatoria. De esta forma se puede observar como el sistema se da cuenta que (en la gran mayoría de 
        los casos) se está intentando acceder a un sector de memoria que es ajeno al proceso que esta corriendo y no lo permitirá. Para ver más acerca de este 
        comando se puede ver su código fuente, en el archivo \textit{/src/shell/rma/rma.c}.\\

        Para demostrar el crecimiento en forma dinámica del \textit{stack} de un proceso implementamos el comando \textit{stacksize}, el cual hace crecer el 
        \textit{stack} del proceso hasta un tamaño determinado (el cual recibe como argumento) o de forma indefinida (si no se le proporciona un tamaño tope).
        Se puede observar en la consola como el comando corre un tiempo hasta que termine abruptamente su ejecución cuando alcanza el tamaño máximo determinado 
        o cuando sobrepasa la máxima capacidad para el \textit{stack} de un proceso. Dicha capacidad esta fijada en 1024 páginas, es decir $4 MB$.\\

        Además, puede observarse en la \textit{Log Terminal} como el sistema va reservando páginas de forma incremental.\\

        Para ver más acerca de este comando se puede ver su código fuente, en el archivo \textit{/src/shell/stacksize/stacksize.c}.\\


        \subsection{Log Terminal}

        Decidimos habilitar una quinta terminal reservada exclusivamente para usos de \textit{logging}. 
        Estos es, el sistema puede dejar registro de ciertas acciones que realiza o errores que ocurran y esta terminal permite al usuario ver este registro. De más
        está decir que está terminal es de sólo lectura, el usuario no puede escribir o realizar cambios.\\

        Decidimos implementar 4 níveles de \textit{logging} y un comando \textit{loglevel} para que el usuario puede elegir el que desee.\\

        \begin{itemize}
        \item QUIET, la terminal no mostrará ningún tipo de registro.
        \item ERROR, se registrarán solo errores que ocurran en el sistema.
        \item INFO, se registrará además información sobre las acciones del sistema.
        \item DEBUG, se registrará también datos sobre depuraciones del sistema.
        \end{itemize}

        La decisión de implementar esta \textit{Log Terminal}, resultó muy útil tanto a la hora de testear y depurar nuestro sistema, como a la hora de demostrar al 
        usuario las distintas funcionalidades del mismo (protección de memoria, cache de disco, stack creciente de forma dinámica, etc...).\\


        \subsection{Memory managment}
        Inicialmente, el trabajo práctico contaba con múltiples arrays de tamaño estático para guardar todos los datos internos del \textit{kernel}.
        Por supuesto, esta solución es menos que óptima, ya que genera un desperdicio de memoria e impone límites arbitrarios al sistema.
        Para evitar esto, desarrollamos un sistema de allocación de páginas de memoria.
        Y encima de este sistema, utilizamos la implementación de \textit{malloc} de Dario Sneidermanis, publicada en \href{https://github.com/esneider/malloc}{Github}.

        Para allocar páginas, hicimos uso de información provista por \textit{Grub} que permite saber exactamente qué zonas de memoria se pueden utilizar, 
        y cuanta memoria hay disponible.
        Esta memoria la dividimos en bloques de $4KB$, teniendo en mente la posibilidad de agregar paginación al sistema.
        Para obtener estas páginas optamos por un simple algoritmo de allocación basado en un bitmap de uso de páginas.

        Una característica interesante de la implementación de \textit{malloc}, es que permite tener varios contextos de memoria distintos.
        Lo que facilitó tener un contexto de memoria para el \textit{kernel}, y otro para cada proceso (excepto los procesos de kernel que usan el mismo 
        que el \textit{kernel}).
        La única limitación del sistema es que los contextos de proceso tienen un tamaño fijo, mientras que el del  \textit{kernel} crece a medida que es necesario.\\


        \subsection{Terminal control sequences}
        Nuestra implementación de terminal soporta una gran parte de \href{http://invisible-island.net/xterm/ctlseqs/ctlseqs.html}{xterm control sequences}.
        Estas permiten manipular el cursor y el color de la pantalla desde los procesos utilizando sólo la primitiva \textit{write}.


\newpage
\section{Problemas encontrados}
\label{sec:Problemas encontrados}

Durante el desarrollo de este trabajo práctico fueron surgiendo diferentes dudas y problemas. El propósito de esta
sección es comentarlos con el fin reflejar proceso de aprendizaje.

        \subsection{Scrolling}
        Dado el alto número de comandos que presenta nuestro sistema operativo, llegó un momento en el que la lista de todos ellos no entraba en una sola pantalla.
        Como implementar scrolling vertical en la terminal presenta un desafio importante, optamos por una solución más simple pero igual de efectiva.
        Los comandos que presentan grandes cantidades de información, como \textit{help} y \textit{ls}, cuando van a escribir más alla de altura de la pantalla, 
        paginan la salida.
        El comportamiendo es muy parecido al de la utilidad \textit{more}, pero se implementó en casos particulares y no como una utilidad genérica. \\


        Como funcionalidad extra, y dado que la forma en que implementamos las cosas nos facilitó en gran parte la tarea, ofrecemos la posibilidad de scrollear una 
        pantalla para arriba o para abajo en todo momento usando las teclas \textit{RePag} y \textit{AvPag} respectivamente.

        \subsection{Argv}
        En su \textit{entry-point} todos los procesos reciben el comando de invocación que los creó (por ejemplo ``cat myexaple'' o ``shell'').
        Originalmente este formaba parte de la estructura de proceso, y simplemente se pasaba una referencia a este campo.
        Pero al activar paginación, y ocultar la memoria donde este dato está contenido nos encontramos con un problema.
        La solución trivial es simplemente hacer accesible esta memoria al proceso, pero va en contra de la utilidad de activar paginación en el primer lugar.
        Finalmente, solucionamos este problema copiando el valor al principio del stack del proceso, y pasando una referencia a este lugar al \textit{entry point}.

	    \subsection{ungetc}

	    La manera en la que implementamos la función de la librería estándar \textit{ungetc} comprendía entre otras cosas la
	    extensión de nuestra estructura \textit{FILE}, conteniendo variables de control para el correcto funcionamiento de la función.
	    Dado que la entrada estándar, la salida estándar y la salida de errores utilizaban estan funciones, en un principio fue		
	    natural implementarlas como variables externas accesibles por cualquier proceso. Con los cambios de este trabajo ya no era
	    aceptable esta implementación.
	    Como solución encontramos que cada proceso contenga dentro de un área reservada de memoria las estructuras \textit{FILE} de 
	    estos flujos de información. Dado que estas estructuras contenían información adicional referente a \textit{ungetc} fue 
	    necesario reservar una página para almacenar esto. De forma arbitraria sea eligió utilizar la dirección lógica en los $3GB$ 
	    de memoria para almacenar esto.

\newpage

\section{Conclusiones}

Durante el desarrollo de este trabajo práctico pudimos sacar distintas conclusiones.

Por un lado, es notable que el sistema operativo tarda un tiempo considerable en comenzar en comparación con la versión previa.
Esto es debido a la activación de la unidad de paginación ya que requiere reservar memoria e inicializar ciertas estructuras, 
incluyendo las tablas de páginas de cada proceso al comenzar. 

Por otra parte, al contrario de las expectativas el uso de una memoria caché no acelera la ejecución sino que la ralentiza. Esto se
debe a que nuestro sistema operativo ya contaba con un capa simil caché a nivel \textit{ext2}. 

Con la adición de paginación y \textit{User-space}, el sistema operativo quedó más completo y robusto.
Si bien quedan algunos problemas de seguridad que solucionar, los procesos ejecutan en contextos totalmente aislados.
Por ejemplo, uno de los problemas por resolver es que utilizando \textit{read} y \textit{write} un proceso de usuario puede leer y escribir cualquier dirección de memoria.
Esto permite por ejemplo acceso a información de otros procesos y ejecución de código arbitrario con permisos de \textit{kernel}.

Con respecto al tema de la caché, cabe remarcar que dado el criterio para el desalojo (\textit{eviction}) escogido, LRU, la estructura de datos que decidimos usar mostró ser una 
excelente decisión dada la flexibilidad que ofrece al poder mantener el orden y poder acceder a un elemento en tiempo constante. La estructura de datos de la que estamos hablando es 
la tabla de punteros a nodos de una lista doblemente linkeada. (Ver sección \ref{Memoria caché}).

Uno de los puntos más fuertes de este trabajo práctico es el hecho de poder manejar los \textit{Page Faults} debido a un crecimiento del \textit{stack}. Esto es gracias a la decisión 
de diseño de tener un \textit{kernel stack} por proceso, destinado al manejo de interrupciones, lo que es posible debido al uso de un \textit{TSS}, como ya se explicó en la sección 
\ref{sec:Proceso}.


\newpage     
\section{Referencias}

Para poder realizar este trabajo, se llevo a cabo un gran proceso de investigación y aprendizaje. Estas son algunas fuentes que resultaron útiles para realizar 
este trabajo.\\

\begin{itemize}
  \item Material provisto por la cátedra
  \item The C programming language - Kernighan y Ritchie
  \item http://invisible-island.net/xterm/ctlseqs/ctlseqs.html
  \item http://webpages.charter.net/danrollins/techhelp/0087.HTM
  \item http://faydoc.tripod.com/cpu/rdtsc.htm
  \item http://stanislavs.org/helppc/
  \item http://www.linux.it/~rubini/docs/ksys/ksys.html
  \item http://wiki.osdev.org
  \item http://wiki.osdev.org/Detecting\_CPU\_Speed
  \item	http://wiki.osdev.org/CMOS\#Accessing\_CMOS\_Registers
  \item http://wiki.osdev.org/Bootable\_CD
  \item http://wiki.osdev.org/Boot\_sequence\#Easy\_Way\_Out
  \item http://wiki.osdev.org/Ext2
  \item http://wiki.osdev.org/IDE
  \item http://wiki.osdev.org/ATA\_PIO\_Mode
  \item http://en.wikipedia.org/wiki/System\_time\#Retrieving\_system\_time
  \item http://en.wikipedia.org/wiki/Calculating\_the\_day\_of\_the\_week
  \item http://cplusplus.com/
  \item http://github.com/esneider/malloc
  \item Unix System Programming, Second Edition - K. Haviland, D. Gray, B. Salama.
  \item http://www.nongnu.org/ext2-doc/ext2.html
  \item http://wiki.osdev.org/Blocking\_Process
  \item http://forum.osdev.org/viewtopic.php?f=1\&t=21719
  \item http://f.osdev.org/viewtopic.php?f=1\&t=18738
  \item http://wiki.osdev.org/Context\_Switching
  \item http://wiki.osdev.org/Scheduling\_Algorithms
  \item http://en.wikipedia.org/wiki/Scheduling\_(computing)
  \item http://en.wikipedia.org/wiki/Round-robin\_scheduling
  \item http://en.wikipedia.org/wiki/Nice\_(Unix)
  \item http://en.wikipedia.org/wiki/Htop
  \item http://en.wikipedia.org/wiki/Top\_(Unix)
  \item http://www.suse.de/~agruen/acl/linux-acls/online/
  \item http://www.gnu.org/s/hello/manual/libc/Users-and-Groups.html
  \item http://www.gnu.org/s/hello/manual/libc/User-and-Group-IDs.html\#User-and-Group-IDs
  \item http://www.gnu.org/s/hello/manual/libc/Process-Persona.html\#Process-Persona
  \item http://www.gnu.org/s/hello/manual/libc/User-Database.html\#User-Database
  \item http://www.gnu.org/s/hello/manual/libc/Group-Database.html\#Group-Database
%Nuevos para este tp3
  \item http://en.wikipedia.org/wiki/Cache
  \item http://en.wikipedia.org/wiki/Cache\_algorithm
  \item http://en.wikipedia.org/wiki/Page\_cache 
  \item Intel IA32 Software Developer vol. 3
\end{itemize}
   
\end{document}
